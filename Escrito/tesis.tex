\documentclass[11pt,a4paper,twoside]{tesis}
% SI NO PENSAS IMPRIMIRLO EN FORMATO LIBRO PODES USAR
%\documentclass[11pt,a4paper]{tesis}

\usepackage{graphicx}
\usepackage[utf8]{inputenc}
\usepackage[spanish]{babel}
\usepackage[left=3cm,right=3cm,bottom=3.5cm,top=3.5cm]{geometry}

\begin{document}

%%%% CARATULA

\def\autor{Autor}
\def\tituloTesis{Título del Trabajo de Especialización}
\def\runtitulo{Resumen}
\def\runtitle{Título del Trabajo}
%\def\director{Obi-Wan Kenobi}
%\def\codirector{Master Yoda}
\def\lugar{Buenos Aires, 2022}
\newcommand{\HRule}{\rule{\linewidth}{0.2mm}}
%
\thispagestyle{empty}

\begin{center}\leavevmode

\vspace{-2cm}

\begin{tabular}{l}
\includegraphics[width=3.6cm]{logouba.png}
\end{tabular}


{\large \sc Universidad de Buenos Aires\\Facultad de Ciencias Exactas y Naturales \\ Facultad de Ingeniería}

\vspace{6.0cm}

%\vspace{3.0cm}
%{
%\Large \color{red}
%\begin{tabular}{|p{2cm}cp{2cm}|}
%\hline
%& Pre-Final Version: \today &\\
%\hline
%\end{tabular}
%}
%\vspace{2.5cm}

\begin{huge}
\textbf{\tituloTesis}
\end{huge}

\vspace{2cm}

{\large Trabajo Final de Especialización en Explotación de Datos y Descubrimiento del Conocimiento}

\vspace{2cm}

{\Large \autor}

\end{center}

\vfill

{\large

%{Director: \director}

\vspace{.2cm}

%{Codirector: \codirector}

\vspace{.2cm}

\lugar
}

\newpage\thispagestyle{empty}


%%%% ABSTRACTS, AGRADECIMIENTOS Y DEDICATORIA
\frontmatter
\pagestyle{empty}
%\begin{center}
%\large \bf \runtitulo
%\end{center}
%\vspace{1cm}
\chapter*{\runtitulo}

\noindent Acá va un resumen del trabajo. Con el resumen se deberia poder tener una idea del trabajo en su totalidad. Desde los obetivos y los datos hasta los resultados y conclusiones. Se escibe al final. (aprox. 200 palabras).

\bigskip

\noindent\textbf{Palabras claves:} representativas del trabajo, los métodos, los datos (no menos de 5).

%\cleardoublepage
%\input{abs_en.tex} % OPCIONAL: comentar si no se quiere

%\cleardoublepage
%\input{agradecimientos.tex} % OPCIONAL: comentar si no se quiere

%\cleardoublepage
%\input{dedicatoria.tex}  % OPCIONAL: comentar si no se quiere

%\cleardoublepage
\tableofcontents

\mainmatter
\pagestyle{headings}

%%%% ACA VA EL CONTENIDO DE LA TESIS

\chapter{Introducción}

\section{Descripción del problema y motivación ($>$ 1/2 carilla)}

Ejemplo de ecuación:
\begin{equation}
L = - \sum_{p \in P} \sum_{y \in Y} w_{py} \sum_{t \in T_{py}} y_t \log \hat y_t + (1-y_t) \log (1-\hat y_t)
\label{eq:loss}
\end{equation}

\section{Trabajos previos ($>$ 1 carilla)}
En la imagen \ref{fig:label} hay un ejemplo de como poner una figura:
\begin{figure}[h]
\centering
\includegraphics[width=0.5\columnwidth]{logouba.png}
\caption{Acá va una explicación de la figura. Recuerden usar colores solo si son explicativos, no olvidar nombrar correctamente los ejes y poner una explicación en el caption.}
\label{fig:label}
\end{figure}

\section{Objetivos ($>$ 1/2 carilla)}

\chapter{Materiales y Métodos}

\section{Datos ($>$ 1 carilla)} 
Descripción de los datos / obligatorio poner la fuente.

\section{Análisis exploratorio}

\chapter{Métodos ($>$ 3 carillas)}
Describir los métodos utilizados (a utilizar). 
Es bligatorio agregar un parrafo especificando a que materías se relacionan.

\section{Método I }
\section{Método II}
\section{Métricas}

\chapter{Experimentos (si los hay)}
Pueden ser comparaciones entre los métodos utilizados, por ejemplo. 

En la \ref{table:tab} se ve un ejemplo de tabla:
\begin{table}[h!]
\centering
\footnotesize
\begin{tabular}{lrrrrrr}
\hline
Phone &  Total & \% Errors &F1 &  EER \\
\hline
   EY &    441 &  13.83 & 0.80 & 0.12 \\
   JH &    178 &  39.89 & 0.81 & 0.18 \\
   AY &   1040 &   5.96 & 0.45 & 0.21 \\
    R &   1298 &  18.34 & 0.53 & 0.27 \\
\hline
\end{tabular}
\caption{Describir la tabla como para que alguien que agarra el trabajo no dependa del cuerpo del texto para entenderla}
\label{table:tab}
\end{table}

\chapter{Resultados ($>$ 3 carillas)}
Pueden ser preliminares. Es bligatorio tener algo hecho.


\chapter{Conclusiones ( $>$ 1 carilla)} 
Pueden ser preliminares. 
Si no hay mucho hecho se pueden discutir las dificultades a futuro.


%%%% BIBLIOGRAFIA
\backmatter
%\bibliography{tesis}

\end{document}
